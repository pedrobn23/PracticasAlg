% Email: pbaeyens31+github@gmail.com
% Licencia: CC BY-SA 3.0

%% Paquetes y configuración %

% Beamer
\PassOptionsToPackage{unicode}{hyperref}  % Evita errores con caracteres no ASCII
\PassOptionsToPackage{naturalnames}{hyperref} % tex.stackexchange.com/questions/10555
\documentclass[compress]{beamer}
\usepackage{adjustbox}
% Idioma
\usepackage[spanish]{babel} % Traducciones
\usepackage[utf8]{inputenc} % Uso de caracteres UTF-8
\usepackage{lmodern}        % Fuentes de tamaño arbitrario
\usepackage[T1]{fontenc}    % Permite copiar y evita errores
%\uselanguage{Spanish}       % Traducciones beamer
%\languagepath{Spanish}      % (tex.stackexchange.com/questions/168208)

\usepackage{multicol}
\usepackage{multirow}

% Matemáticas
\usepackage{amsfonts}
\usepackage{amsmath}
\usepackage{amssymb}
\usepackage{marginnote}
% Colores
\definecolor{backg}{HTML}{F2F2F2}    % Fondo
\definecolor{title}{HTML}{bdc3d1}    % Títulos
\definecolor{comments}{HTML}{BDBDBD} % Comentarios
\definecolor{keywords}{HTML}{08388c} % Palabras clave
\definecolor{strings}{HTML}{FA5858}  % Strings
\definecolor{links}{HTML}{2C2C95}    % Enlaces
\definecolor{bars}{HTML}{045FB4}     % Barras (gráfico)

% Código
\usepackage{listings}
\lstset{
language=[LaTeX]TeX,
basicstyle=\footnotesize,
morekeywords={href,uselanguage,languagepath,column},
otherkeywords={pause,usetheme,usecolortheme,useinnertheme,titlepage,tableofcontents,subtitle},
breaklines=true,
backgroundcolor=\color{backg},
keywordstyle=\color{keywords},
commentstyle=\color{comments},
stringstyle=\color{strings},
tabsize=2,
% Acentos, ñ, ¿, ¡ (tex.stackexchange.com/questions/24528)
extendedchars=true,
literate={á}{{\'a}}1 {é}{{\'e}}1 {í}{{\'i}}1 {ó}{{\'o}}1
         {ú}{{\'u}}1 {ñ}{{\~n}}1 {¡}{{\textexclamdown}}1
         {¿}{{?`}}1
}

% Gráficos
\usepackage{pgfplots}
\pgfplotsset{width=7cm,compat=1.14} % Opciones para gráficos

% Emoticonos
\usepackage{wasysym}
\usepackage{graphicx}
% tikz
\usepackage{tikz}
\usetikzlibrary{mindmap,trees,shadows}
\tikzset{ % Genera overlays
    invisible/.style={opacity=0},
    visible on/.style={alt={#1{}{invisible}}},
    alt/.code args={<#1>#2#3}{\alt<#1>{\pgfkeysalso{#2}}{\pgfkeysalso{#3}}},
}

%% Comandos %%
\newcommand{\ejemplo}[1]{\lstinputlisting{./examples/#1}} % Mostrar código de ejemplos
\newcommand{\muestra}[1]{\input{./examples/#1}}           % Mostrar ejemplos
\newcommand{\seccion}[1]{\input{./sections/#1}}           % Incluir secciones
\newcommand{\espacio}{\vspace*{\baselineskip}}            % Añade espacios
\newcommand{\beamer}{\texttt{beamer} }                    % Estilo único para beamer
\newcommand{\enlace}[3]{\href{#1}{\textbf{#2}} - {\small #3}}  % Estílo único para refs
\newcommand{\comando}[1]{{\color{black}\textbackslash}{\color{keywords}#1}}
\newcommand{\marginalnote}[1]{\mbox{}\marginpar{\raggedright\hspace{0pt}#1}}
%% Temas %%
% Tema y tema de color
  \usetheme{Dresden}
  \usecolortheme{dolphin}
  \useinnertheme{circles}
  \setbeamercovered{transparent}
% Colores bloques
  \setbeamercolor{block title}{bg=title,fg=links}
  \setbeamercolor{block body}{bg=backg,fg=black}
  \setbeamercolor{block title alerted}{fg=red!70!black,bg=title!92!red}
  \setbeamercolor{block body alerted}{fg=black,bg=backg}
  \setbeamercolor{block title example}{fg=green!70!black,bg=title!92!green}
  \setbeamercolor{block body example}{fg=black,bg=backg}
% Enlaces (tex.stackexchange.com/questions/13423)
\hypersetup{colorlinks,linkcolor=,urlcolor=links}
% Quita enlaces de navegación (stackoverflow.com/questions/3017030)
\setbeamertemplate{navigation symbols}{}
% Quita barra inferior (stackoverflow.com/questions/1435837)
\setbeamertemplate{footline}{}
% Evita warnings boxes
\hfuzz=20pt
\vfuzz=20pt
% Evita wranings itemize
\renewcommand\textbullet{\ensuremath{\bullet}}

%% Título y otros %%
\title{Práctica 3}                                               % Título
\subtitle{Algoritmos \textit{greedy}}                                  % Subtítulo
\author{Laura Gómez Garrido, Pedro Bonilla Nadal, Javier Sáez Maldonado, Daniel Pozo Escalona, Luis Ortega Andrés}

\date{2º Doble Grado Informática y Matemáticas}                                                            % Fecha

%% Listing settings

\usepackage{color}

\definecolor{dkgreen}{rgb}{0,0.6,0}
\definecolor{gray}{rgb}{0.5,0.5,0.5}
\definecolor{mauve}{rgb}{0.58,0,0.82}

\lstset{frame=tb,
  language=Java,
  aboveskip=3mm,
  belowskip=3mm,
  showstringspaces=false,
  columns=flexible,
  basicstyle={\small\ttfamily},
  numbers=none,
  numberstyle=\tiny\color{gray},
  keywordstyle=\color{blue},
  commentstyle=\color{dkgreen},
  stringstyle=\color{mauve},
  breaklines=true,
  breakatwhitespace=true,
  tabsize=3
}

%% Presentación %%
\begin{document}

\begin{frame}
\titlepage
\end{frame}

\begin{frame}{Índice}
  \hypertarget{index}{}
  \tableofcontents
  
\end{frame}

\section{Problema}
\begin{frame}{Problema}

	Se pide diseñar e implementar un algoritmo \textit{greedy} que solucione el problema de encontrar la triangulación mínima de un polígono convexo. Como salida, se deberá proporcionar el conjunto de aristas que forman la triangulación junto con su coste (suma de las longitudes de las cuerdas generadas).

\end{frame}	

\section{Conceptos}
\begin{frame}{Conceptos}
	
\begin{block}{Polígono convexo}
Un polígono del plano se dice \textbf{convexo} si cada uno de sus vértices es un vértice de su envolvente convexa.
\end{block}
\pause
\begin{block}{Triangulación}
 Por \textbf{triangulación} de un polígono plano entenderemos la división de este en un conjunto de triángulos, con la restricción de que dos lados de dos triángulos distintos no se corten o lo hagan en un vértice del polígono.
\end{block}
\pause
\begin{block}{Triangulación mínima}
Una triangulación se dice \textbf{mínima} si es un elemento minimal del conjunto de triangulaciones posibles, con el preorden inducido por el orden en $\mathbb{R}$ de la suma de las longitudes de sus diagonales.
\end{block}

\end{frame}


\begin{frame}{Ejemplo de polígono triangulado}
\begin{center}

\begin{tikzpicture}[scale=2]
\draw (1,1) -- (2,1);
\draw (2,1) -- (3,2);
\draw (3,2) -- (2,3);
\draw (2,3) -- (1.5,3);
\draw (1.5,3) -- (0.7,2);
\draw (0.7,2) -- (1,1);

\draw[color=blue] (1,1) -- (3,2);
\draw[color=blue] (2,3) -- (0.7,2);
\draw[color=blue] (3,2) -- (0.7,2);

\end{tikzpicture}
\end{center}

\end{frame}

\section{Solución}

\subsection{Descripción del algoritmo}

\begin{frame}{Descripción del algoritmo}
	La entrada del algoritmo es un conjunto de $n$ puntos del plano euclídeo, los vértices del polígono convexo que queremos triangular, ordenados. De esta forma, los lados del polígono quedan determinados por el orden, es decir, son: 
\[\mathcal{L} = \{\alpha_i \vee \alpha_{i+1}\ :  i\in \mathbb{Z}_n\}\]

Del mismo modo, las cuerdas son:
\[\mathcal{C} = \{\alpha_i \vee \alpha_{i+2}\ : i\in \mathbb{Z}_n\}\]
\end{frame}

\begin{frame}{Descripción del algoritmo}

\begin{enumerate}
    \item Si el polígono es un triángulo, no se hace nada.
	\item Encontrar una cuerda de longitud mínima.
	\item Almacenar la cuerda en el conjunto solución.
	\item Eliminar del conjunto de vértices el comprendido entre los dos vértices que une la cuerda.
	\item Ejecutar el algoritmo desde el primer paso, siendo esta vez la entrada el conjunto de vértices resultante en el paso anterior.
\end{enumerate}

\end{frame}

\begin{frame}{Elementos del algoritmo \textit{greedy}}
\begin{itemize}
\item Lista de candidatos: son todas las cuerdas del polígono que se está evaluando en cada paso.                                     

\item Lista de candidatos utilizados: diagonales (cuerdas o no) del primer polígono que han sido seleccionadas.                                                                         

\item Función solución: el polígono tiene tres vértices.

\item Criterio de factibilidad: el predicado $0$-ario verdadero.

\item Función de selección: escogemos una cuerda de longitud mínima.                      

\item Función objetivo: la suma de las longitudes de las cuerdas. Queremos minimizarla.
\end{itemize}

\end{frame}

\begin{frame}{Eficacia del algoritmo}
El algoritmo no es óptimo, en el sentido de que no siempre encuentra una triangulación mínima, aunque lo hace en la mayoría de los casos o se aproxima muy bien.
\end{frame}

\begin{frame}{}
\begin{columns}[c] 
    \column{.5\textwidth}
     \begin{tikzpicture}
%% \node at (0.936153,1.76738) {A};
%% \node at(-1.12177,1.65579) {B};
%% \node at(-1.3,-1.51987) {C};
%% \node at(-1.03769,-1.70974) {D};
%% \node at(0.911164,-1.78039) {E};
%% \node at(1.65839,-1.11792) {F};
%% \node at(1.99906,-0.0612251) {G};
\draw (0.936153,1.76738) -- (-1.12177,1.65579);
\draw (-1.12177,1.65579) -- (-1.3,-1.51987);
\draw (-1.3,-1.51987) -- (-1.03769,-1.70974);
\draw (-1.03769,-1.70974) -- (0.911164,-1.78039);
\draw( 0.911164,-1.78039) -- (1.65839,-1.11792);
\draw( 1.65839,-1.11792) -- (1.99906,-0.0612251);
\draw (1.99906,-0.0612251) -- (0.936153,1.76738);

\end{tikzpicture}

    \column{.5\textwidth}
(0.936153,1.76738)
(-1.12177,1.65579)\\
(-1.3,-1.51987)\\
(-1.03769,-1.70974)\\
(0.911164,-1.78039)\\
(1.65839,-1.11792)\\
(1.99906,-0.0612251)

    \end{columns}
\hfill\\
\hfill\\
En este ejemplo, el peso de la triangulación \textit{greedy} es $11.7845$, mientras que existe una de peso $11.43$.
\end{frame}


\subsection{Implementación}
\begin{frame}[fragile]
	\frametitle{Avance circular de un iterador}
 \begin{lstlisting}
auto circular_advance = [](auto& forwdIt, auto initValue, auto endValue, int n) {
	for(int i=0; i<n; i++)
	{
		forwdIt++;
		if(forwdIt == endValue)
			forwdIt = initValue;
	}

	return forwdIt;
};
 \end{lstlisting}
 
Esta es una función auxiliar, cuyo propósito es implementar la aritmética de un anillo de restos sobre un contenedor con un iterador que soporte el operador de incremento (\texttt{++}).
 
\end{frame}

\begin{frame}[fragile]
	\frametitle{Cuerda de longitud mínima}
        \begin{adjustbox}{width=10cm}
        
\begin{lstlisting}[basicstyle= \tiny,]
auto find_min_string = [&min_distance, &it_min, &min_string, &points](auto initial_point_it) {
    auto p0 = initial_point_it;
    auto p = p0;
    
    do {
        auto prev = p;
        
        circular_advance(p, points.begin(), points.end(), 2);
        if(min_distance > euclidean_distance(*prev, *p))
        {
            min_distance = euclidean_distance(*prev, *p);
            min_string = std::make_pair(*prev, *p);
            it_min = circular_advance(prev, points.begin(), points.end(), 1);
        }
      
    }while(p != p0);
};
\end{lstlisting}
\end{adjustbox}\hfill\\\hfill\\
	Busca una cuerda de mínima longitud recorriendo los nodos hasta volver al nodo inicial.
\end{frame}

\begin{frame}
Si identificamos los vértices $\alpha_1,\dots,\alpha_n$ del polígono con los elementos de $\mathbb{Z}_n$, y observamos:
\begin{itemize}
\item Si $n$ es par $\implies \mathbb{Z}_n/2\mathbb{Z}_n \simeq \mathbb{F}_2$
\item Si $n$ es impar $\implies \mathbb{Z}_n/2\mathbb{Z}_n \simeq \mathbb{F}_1$
\end{itemize}

Podemos concluir que un polígono con un número par de vértices tiene dos ciclos de cuerdas, y uno con un número impar de vértices, uno solo.

Por tanto, para comparar las longitudes de todas las cuerdas de un polígono, debe llamarse a la función anterior dos veces en el primer caso, y una en el segundo.
\end{frame}

\begin{frame}[fragile]
	\frametitle{Bucle principal}
	\begin{lstlisting}
	while(points.size() > 3)
	{
		auto it = points.begin();
		// Unconditionally find minimum length string starting
		// at the first element
		find_min_string(it);
	
		// If the polygon's number of edges is even, we need to
		// do the same starting at the second one
		if(!(points.size()%2)) find_min_string(++it);
	
		sol.push_back(min_string);
		points.erase(it_min);
	}
	\end{lstlisting}
	
\end{frame}

\subsection{La complejidad del algoritmo}
\begin{frame}{La complejidad del algoritmo}


Es un algoritmo de complejidad $O(n^2)$, dado que para un polígono de $n$ nodos, en cada iteración del bucle \texttt{while} ($n$ iteraciones) llama a la función \textit{find\_min\_string}, de complejidad $O(n)$, para después eliminar un vértice.

\[
T(n) = \sum_{i=3}^n ki = k\frac{n^2+n}{2} - 3k \in O(n^2)
\]

donde:
\begin{itemize}
\item $T(n)$ es el número de instrucciones que ejecuta el algoritmo para un polígono de $n$ vértices.
\item $k$ es una constante de proporcionalidad aproximada entre el número de instrucciones que se ejecutan para encontrar la cuerda mínima entre $i$ e $i$. 

\end{itemize}
\end{frame}

\begin{frame}{Alternativa}
Existen implementaciones de este algoritmo que se ejecutan en tiempo $O(n\log(n))$, algunas están descritas en:
\hfill\\\hfill\\
Dickerson, Matthew T. (1997). \textit{«Fast greedy triangulation algorithms»}. Computational Geometry (Elsevier)  67-86
\end{frame}

\end{document}
